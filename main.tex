\documentclass[conference]{IEEEtran}

\usepackage{graphicx}

\begin{document}
	
	\title{Chinese Borders}
	
	\author{\IEEEauthorblockN{Ida Hönigmann}
		\IEEEauthorblockA{Technical Secondary College\\
			Department of Computer Science\\
			2700 Wiener Neustadt, Austria\\
			Email: hoenigmann.ida@student.htlwrn.ac.at}
	}
	
	\maketitle
	
	\begin{abstract}
		The abstract goes here.
	\end{abstract}
	
	\section{Introduction}
	As China is not particularly friendly to most of its neighbouring countries and has the longest land border in the world, it is no wonder that there are several disputes and even more interesting facts about these lines on the map.
	
	The topic of Chinese borders gives an insight into Chinese history, for which the Russian and Mongolian relations to China as well as Hong Kong and Macau are examples. It explores who and what gets transported out and into the country, such as workers building the Belt and Road Initiative, products being sold by China and North Koreans fleeing their country. Border disputes and the people living in these regions are yet another interesting case study along the Chinese border to Pakistan, India, Bhutan, Myanmar and, although already resolved, Nepal. While one would think being a communist lead country would help increase the relation to China, as it did with Laos, Vietnam shows that China is more picky and looks for more than just political beliefs. After having established all these interesting relations to all of its bordering countries China on land it tried extending its influence on the South China Sea. 
	
	\section{North Korea}
	
	Along the border separating China from North Korea lies one of the holy sights of all Koreans: Mount Paektu. \cite{theIndianExpress_explainedWhatIsTheSignificanceOfMtPeaktuForKinJongUn} Here South Koreans as well as North Koreans visit the Heaven Lake, although separated by multiple meters, as to not be able to see one another. From this mountain the two rivers Yalu and Tumen form. These rivers serve as the separation between the two countries.
	
	Further west in Dandong a significant portion of North Korea's trade with the outside world is being moved by trucks over a single bridge. These trucks are being loaded up in China and send to North Korea, where they are unloaded and send back - mostly empty. The imports from China make up about 57\%\cite{wp_economyOfNorthKorea} of Norths Korea's imports in total.
	
	One of the reasons China is exporting so many goods to North Korea is fear of the regime collapse, which would send thousands of North Korean emigrants over the border to China.
	
	As there are already some defectors crossing the border further east, the shallow river is protected by Chinese guards.\cite{yp_anInconvenientBorderWhereChinaMeetsNorthKoreaABCNews} All fleeing persons found, are send back to North Korea, where they will most likely face death.
	
	\section{Russia}
	
	\section{Mongolia}
	
	\section{Kazakhstan}
	
	\section{Kyrgyzstan}
	
	\section{Tajikistan}
	
	\section{Afghanistan}
	
	\section{Pakistan}
	
	\section{India}
	
	\section{Nepal}
	
	\section{Bangladesh}
	
	\section{Bhutan}
	
	\section{Myanmar}
	
	\section{Laos}
	
	\section{Vietnam}
	
	\section{Macau}
	
	\section{Hong Kong}
	
	\section{South China Sea}
	
	\begin{figure}[t]
		\centering
		\includegraphics[width=2.5in]{img/image.png}
		\caption{Image caption.}
		\label{pic:image}
	\end{figure}
	
	\section{Conclusion}
	The conclusion goes here.
	
	\section*{Acknowledgment}
	The authors would like to thank...
	
	\bibliographystyle{plain}
	\bibliography{main}
\end{document}


